\documentclass[12pt]{article}
\usepackage[margin=1in]{geometry} 
\usepackage{amsmath}
\usepackage{tcolorbox}
\usepackage{amssymb}
\usepackage{amsthm}
\usepackage{lastpage}
\usepackage{fancyhdr}
\usepackage{accents}
\pagestyle{fancy}
\setlength{\headheight}{40pt}


\newenvironment{solution}
  {\renewcommand\qedsymbol{$\blacksquare$}
  \begin{proof}[Solution]}
  {\end{proof}}
\renewcommand\qedsymbol{$\blacksquare$}

\newcommand{\ubar}[1]{\underaccent{\bar}{#1}} % add packages, settings, and declarations in settings.tex
\usepackage{url,hyperref}
\begin{document}

\lhead{Prof. C.E. Tsourakakis} 
\rhead{CS365 Spring '23 \\ Foundations of Data Science \\ Assignment 1} 
\cfoot{\thepage\ of \pageref{LastPage}}


\section*{Instructions}
\framebox{%
	\begin{minipage}{0.9\linewidth}
		\begin{itemize}
			\item The homework is due on \underline{{\bf Friday 2/3 at 5pm ET}}.  
			\item There are 3 paper-pencil-like problems and 1 coding problem. 
            \item LaTeX is recommended but hand-written solutions will be accepted as long as they are legible. The coding part and the associated text should be coded in the Jupyter notebook provided to you on the class Git repo. 
			\item No extension will be provided, unless for serious documented reasons.
			\item {\bf Start early!}
			\item Study the material taught in class, and feel free to do so in small groups, but the solutions should be a product of your own work. 
			\item This  is not a multiple choice homework;   reasoning, and mathematical proofs are required before giving your final answer.
		\end{itemize}
\end{minipage}}

Name: Caiwei Zahng, BUID: U97031910

\section{Is there a path from $s$ to $t$? [15 points]}  
 \begin{figure*}[!ht]
	\centering
	\includegraphics[width=0.5\textwidth]{pipes}
	\caption{\label{fig:pipes} How likely is water to reach $t$ from $s$?}
\end{figure*}

Consider the network shown in Figure~\ref{fig:pipes}. Recall that $p_i$ is the probability pipe $i$ breaks down, and that pipes break down independently. 



\begin{tcolorbox}
	\begin{enumerate}
		\item What is the probability water can go from the water source $s$ to the destination village $t$? Explain your answer. You do not need to simplify it algebraically.
	%	\item Let $p_i=10^{-3}$ for $i=1,..,5$. Compare your answer from (1) to Benforroni's inequalities for $m=1,2$.  
	\end{enumerate}
\end{tcolorbox}

\subsection{Answer}

\newpage 

\section{PIN Cracker [30 Points] } 
John has a cell phone with a PIN that consists of 4 digits (0-9). Unfortunately John totally forgot his PIN, but at least he can try PIN numbers as many times as he wants to without blocking the device.  He applies the following two strategies\footnote{In reality, there exist better strategies to break a PIN \url{https://www.popsci.com/technology/article/2012-09/infographic-day-fastest-way-crack-4-digit-pin-number/}, but let's assume we employ only naive strategies here.}:

\begin{itemize}
    \item[($s_1$)] He tries a valid PIN uniformly at random each hour, till he enters the right one.
    \item[($s_2$)] He keeps track of the unsuccessful attempts, and chooses a PIN uniformly at random from the PIN numbers he has not tried yet. 
\end{itemize}

\begin{tcolorbox}
\begin{enumerate}
    \item  {\bf [10 points] } Write code that simulates strategies $s_1$ and $s_2$. You may assume that the correct PIN is 2022 in your code. Simulate each strategy 100, 200, 300, ..., 1\,000 times, and report for each number of trials the average number of trials, and the standard deviation till John figures out his PIN. Present your empirical findings in two plots (one per strategy) with error bars, where the $x$-axis is the number of trials.  
    \item  Let $X_1,X_2$ be the number of trials under strategies $s_1$ and $s_2$ respectively. Compute their expectations analytically: 
    \begin{itemize}
        \item[i)]   {\bf [5 points] }$ \mathbb{E}[X_1]$.
        \item[ii)]  {\bf [5 points] } $ \mathbb{E}[X_2]$.
    \end{itemize} 


\item  {\bf [10 points] } Apply Markov's inequality to upper bound the probability $\mathbb{P}(X_i \geq 7\,000)$ for $i=1,2$.  Does Markov's inequality always provide meaningful bounds?
\end{enumerate}
\end{tcolorbox}


\subsection{Answer}
\newpage 

\section{ [25 points]} 
 

Provide clean proofs and calculations for the following problems. 

\begin{tcolorbox}
	\begin{enumerate}
		\item  (5 points)  Two fair dice are rolled. Show that the event that their sum is 7 is independent of the score shown by the first die.  
         \item  (10 points)    For random variables $X,Y$, prove $\mathbb{E}(XY)^2 \leq \mathbb{E}(X^2)\mathbb{E}(Y^2)$. When does equality hold?
    
         \item  (5 points)   If one picks a numerical entry at random from tax documents, the first two significant digits, $X, Y$ respectively, are found to have approximately the joint mass function 

         $$ f(x,y) = \log_{10}(1+\frac{1}{10x+y}), 1\leq x\leq 9, 0\leq y\leq 9.$$

         Find the mass function of the first significant digit $X$. What do you observe, i.e., are digits 1 and 9 equally likely to be the first digit? What is the expected value of $X$?  

         \item  (5 points)  Find the conditional density function  of $Y$ given $X$ when they have the joint density function $f(x,y)=\lambda^2 e^{-\lambda y}$ for $0\leq x \leq y<\infty$. 
	\end{enumerate}
\end{tcolorbox}

 \subsection{Answer}
 Two events A and B are independent if and only if $P(A|B) = P(A)$\\
 Rolling two dice, let A be the event that the sum of the two dice is 7 and
 B be the event that the first die shows a score $x$. \\
 Therefore, $P(A) = 6/36 = 1/6$ and since the second die can only show one score to get a sum of 7,
 $P(A \vert B = x) = \frac{P(A\cap B)}{P(B)} = \frac{1/36}{1/6} =1/6$, where $x$ can be 1,2,3,4,5,6.
 In this way, $P(A) = P(A \vert B)$, which proves that the event that the sum of two dice being 7 is
 independent of the score shown by the first die.

 \subsection{Answer}
 Based on Cauchy-Schwarz inequality, for any two vectors x and y in a Euclidean space,
 $(x\cdot y)^2 \leq x^2 y^2$. Now considering $X, Y$ as two vectors, 
 and the dot product of X and Y is given by the expected value of $XY$. Therefore,
 $(\mathbb{E}(XY))^2 <= (\mathbb{E}(X^2))(\mathbb{E}(Y^2))$. \\
 The equality holds when $X$ and $Y$ are linearly dependent, 
 which means that there exists a non-zero constant c such that $Y = cX$ .

 \subsection{Answer}
 Using Marginal Probability Mass Function of $X$, $f_X{(x)} = \sum_{y} f(x,y)$ as $0\leq y\leq 9$ .\\
 $$f_X{(x)} = log_{10}(1+\frac{1}{10x}) + log_{10}(1+\frac{1}{10x+1}) + \cdots  + log_{10}(1+\frac{1}{10x+9})$$
 $$ f_X{(x)}= log_{10}(\frac{10x+1}{10x} \cdot \frac{10x+2}{10x+1} \dots \cdots \dots  \frac{10x+10}{10x+9} )$$
 $$ f_X{(x)}= log_{10}(\frac{10x+10}{10x})$$
 $$ f_X{(x)}= log_{10}(\frac{x+1}{x})$$
To see whether digits 1 and 9 are equally likely to be the first digit, we just need to compare $f_X{(1)}$ and $f_X{(9)}$.\\
Based on the formula above, $f_X{(1)} = log_{10}(2)$, while $ f_X{(9)}=log_{10}(\frac{10}{9})$, which is not equal.
Therefore we can conclude that they are not equally likely to be the first digit. \\
For expected value of X, we can use density function that $\mathbb{E}(X) = \sum_{x} x f_X{(x)}$, for $1\leq x\leq 9$. \\
$$\mathbb{E}(X) = \sum_{1}^{9} x log_{10} (\frac{x+1}{x})$$
$$\mathbb{E}(X) = 1 log_{10} (\frac{2}{1}) + 2 log_{10} (\frac{3}{2}) + 3 log_{10} (\frac{4}{3}) + \cdots + 9  log_{10} (\frac{10}{9}) $$
$$\mathbb{E}(X) = log_{10}(2) - log_{10}(1) + 2log_{10}(3) - 2log_{10}(2) + 3log_{10}(4) - 3log_{10}(3) + \cdots + 9log_{10}(10) - 9log_{10}(9) $$
$$\mathbb{E}(X) = -log_{10}(1) - log_{10}(2) - log_{10}(3) - \cdots - log_{10}(9) + 9log_{10}(10)$$
$$\mathbb{E}(X) = 9- log_{10} (1 \cdot 2 \cdot 3 \cdots 9)$$
$$\mathbb{E}(X) = 9 - 5.5524 = 3.4476.$$
Therefore, the expected value of first digits is about 3.4476.



 \subsection{Answer}
 Conditional density formula: $f_{Y|X}(y|x) = \frac{f(x,y)}{f_X(x)}$, 
 where $f_X(x)$ is the marginal density function of $X$: $f_X(x) = \int_{x}^{\infty} f(x,y)dy$.\\
 Substituting the given joint density function $f(x,y)=\lambda^2 e^{-\lambda y}$ for $0\leq x \leq y<\infty$ into the expression:
 $$f_{Y|X}(y|x) = \frac{\lambda^2 e^{-\lambda y}}{\int_{x}^{\infty} \lambda^2 e^{-\lambda y}dy}$$
 $$f_{Y|X}(y|x) = \frac{\lambda^2 e^{-\lambda y}}{\frac{\lambda^2}{\lambda} e^{-\lambda x}}$$
 $$f_{Y|X}(y|x) = \lambda e^{\lambda x -\lambda y} $$
 So the conditional density function of $Y$ given $X$ is $\lambda e^{\lambda x -\lambda y} $









  \section{Coding [30 points]}

  The coding part of the homework is available on Git \href{https://github.com/tsourolampis/cs365-spring23/tree/main/hw1}{here}\footnote{https://github.com/tsourolampis/cs365-spring23/blob/main/hw1/HW1coding.ipynb}. Follow the instructions in the Jupyter notebook. After finishing it, download the notebook in \textbf{PDF format}, which should include all the code and figures. If you have trouble converting it to PDF, you can download it as HTML, and then save it as PDF. Submit the math problems and coding problems \textbf{separately in two files} through Gradescope.
\end{document}
